\newcommand{\psd}[1]{{\small\sffamily{\color{blue!60}#1}}}

\subsection{Advance exercise  1}

There is a solver run level flag for mesh refinement
\footnote{Mesh refinement is performed after partitioning.}. This flag
is called \psd{-split [int]} which splits the triangles (resp.
tetrahedrons) of your mesh into four smaller triangles (resp.
tetrahedrons). As such \psd{-split 2} will produce a mesh with 4 times
the elements of the input mesh. Similarly, \psd{-split n} where \(n\) is
a positive integer produces \(2^n\) times more elements than the input
mesh. You are encouraged to use this \psd{-split} flag to produce
refined meshes and check, mesh convergence of a problem, computational
time, etc. Use of parallel computing is recommended. You could try it
out with \psd{PSD\_Solve} or \psd{PSD\_Solve\_Seq}, for example:

\begin{lstlisting}[style=BashInputStyle]
PSD_Solve -np 4 Main.edp -mesh ./../Meshes/2D/bar.msh -v 0 -split 2
\end{lstlisting}

for splitting each triangle of the mesh \psd{bar.msh} into 4.

\subsection{Advance exercise  2}

There is a preprocess level flag \psd{-debug}, which as the name
suggests should be used for debug proposes by developers. However, this
flag will activate OpenGL live visualization of the problems
displacement field. You are encouraged to try it out

\begin{lstlisting}[style=BashInputStyle]
PSD_PreProcess -problem linear_elasticity -dimension 2 -bodyforceconditions 1 \
-dirichletconditions 1 -postprocess u -timelog -debug
\end{lstlisting}

Then to run the problem we need additional \psd{-wg} flag

\begin{lstlisting}[style=BashInputStyle]
PSD_Solve -np 4 Main.edp -mesh ./../Meshes/2D/bar.msh -v 0 -wg
\end{lstlisting}

\subsection{Advance Exercise  3}

One interesting way of solving a linear Elasticity problem is to solve
it via a pseudo nonlinear model. There is a preprocess level flag
\psd{-model pseudo\_nonlinear}, which introduces pseudo nonlinearity
into the finite element variational formulation of linear elasticity.
You are encouraged to use this flag and see how the solver performs.
Indeed, now you should see some nonlinear iterations (1 or 2) are taken
for convergence.

\begin{lstlisting}[style=BashInputStyle]
PSD_PreProcess -problem linear_elasticity -dimension 2 -bodyforceconditions 1 \
-dirichletconditions 1 -postprocess u -timelog -model pseudo_nonlinear
\end{lstlisting}

Then to run the problem

\begin{lstlisting}[style=BashInputStyle]
PSD_Solve -np 4 Main.edp -mesh ./../Meshes/2D/bar.msh -v 0
\end{lstlisting}

To understand what the flag does, try to find out the difference between
the files created by \psd{PSD\_PreProcess} when used with and without
\psd{-model pseudo\_nonlinear} flag. Especially, compare
\psd{LinearFormBuilderAndSolver.edp} and
\psd{VariationalFormulations.edp} files produced by
\psd{PSD\_PreProcess} step. You will see Newton--Raphsons iterations are
performed for solving the linear problem. However, the nonlinear
iterations loop converges very rapidly (in 1 iteration) due to linear
nature of the problem. \textbf{Note:} This flag is exclusive for
parallel solver.

\subsection{Advance exercise 4}

There is a preprocess level flag \psd{-withmaterialtensor}, which
introduces the full material tensor into the finite element variational
formulation. You are encouraged to use this flag and see how the solver
performs.

\begin{lstlisting}[style=BashInputStyle]
PSD_PreProcess -problem linear_elasticity -dimension 2 -bodyforceconditions 1 \
-dirichletconditions 1 -postprocess u -timelog -withmaterialtensor
\end{lstlisting}

Then to run the problem

\begin{lstlisting}[style=BashInputStyle]
PSD_Solve -np 4 Main.edp -mesh ./../Meshes/2D/bar.msh -v 0
\end{lstlisting}

To understand what the flag does, try to find out the difference between
the files created by \psd{PSD\_PreProcess} when used with and without
\psd{-withmaterialtensor} flag. Especially, compare
\psd{FemParameters.edp}, \psd{MeshAndFeSpace} and
\psd{VariationalFormulations.edp} files produced by
\psd{PSD\_PreProcess} step.
