\newcommand{\psd}[1]{{\small\sffamily{\color{blue!60}#1}}}

Similar simulation, as in the previous tutorial is presented in this
section. We showcase the 2D bar problem simulation with one end clamped
wile being pulled at the other end. Just like the previous simulation
the body force is neglected. However, now the non clamped ends pull is
approximated with Neumann force (aka. traction force)
\(\int_{\partial\Omega^h_{\text N}}(\mathbf t \cdot \mathbf{v}^h)\). To
simulate the pull we assume traction vector
\(\mathbf t=[t_x,t_y]=[10^9.,0]\) acting on the non clamped right end of
the bar, i.e., force in \(x\) direction is 10\(^9\) units. Here is how
PSD simulation of this case can be performed. The same problem from
previous tutorials 1 and 2 is used here, a bar 5 m in length and 1 m in
width, and is supposed to be made up of a material with density
\(\rho=8\times 10^3\), Youngs modulus \(E=200\times 10^9\), and Poissons
ratio \(\nu=0.3\).

\textbf{Step 1: Preprocessing}

First step in a PSD simulation is PSD preprocessing, at this step you
tell PSD what kind of physics, boundary conditions, approximations,
mesh, etc are you expecting to solve.

In the terminal \psd{cd} to the folder
\psd{/home/PSD-tutorials/linear-elasticity}. Launch
\psd{PSD\_PreProcess} from the terminal, to do so run the following
command.

\begin{lstlisting}[style=BashInputStyle]
PSD_PreProcess -problem linear-elasticity -dimension 2 -dirichletconditions 1 -tractionconditions 1 -postprocess u
\end{lstlisting}

the comandline flag \psd{ -dirichletconditions 1}, notifies to PSD that
there is one Dirichlet border ---the clamped end of the bar--- in this
simulation. And the flag \psd{ -tractionconditions 1} notifies to PSD
that there is one traction border ---the right end of the bar--- in this
simulation.

To provide the clamped boundary condition (\(u_1=0,u_2=0\)) set the
variables \psd{ Dbc0On 2}, \psd{ Dbc0Ux 0.}, and \psd{ Dbc0Uy 0.} in
\psd{ ControlParameters.edp}. In the same file traction boundary
conditions are provided via the variables \psd{ Tbc0On 4} and
\psd{ Tbc0Tx 1.e9}, which mean apply traction force
\(\mathbf t=[t_x,t_y]=[10^9.,0]\) on label number 4 (right) of the mesh.
If user wishes to add traction force ,for instance \(t_y=100.\), simply
add the missing macro \psd{ macro Tbc0Tx 1.e9 //}.

\textbf{Step 2: Solving}

Let us now use 5 cores to solve this problem. To do so enter the
following command:

\begin{lstlisting}[style=BashInputStyle]
PSD_Solve -np 5 Main.edp -mesh ./../Meshes/2D/bar.msh -v 0
\end{lstlisting}

Notice, that this is the exact same command used in solving the previous
bar problems from other sections, with only difference that we now use
\psd{ -np 5}.

Note that for this simple problem, the bar mesh (\psd{bar.msh}) has been
provided in \psd{../Meshes/2D/"} folder, this mesh is a triangular mesh
produced with Gmsh. Moreover detailing meshing procedure is not the
propose of PSD tutorials. A user has the choice of performing their own
meshing step and providing them to PSD in
\psd{.msh}\footnote{Please use version 2} or \psd{.mesh} format, we
recommend using Salome or Gmsh meshers for creating your own geometry
and meshing them.

\textbf{Step 3: Postprocessing}

Launch ParaView and have a look at the \psd{ .pvd} file in the
\psd{ PSD/Solver/VTUs\_DATE\_TIME} folder.

\begin{figure}[htbp]
    \centering
    \begin{minipage}[t][2cm][t]{0.36\textwidth}
    \includegraphics[align=b,width=1\textwidth]{./Images/2d-bar-partitioned5.png}
    \end{minipage}\hspace{.1\textwidth}
    \begin{minipage}[t][2cm][t]{0.5\textwidth}
    \includegraphics[align=b,width=1\textwidth]{./Images/2d-bar-clamped-traction.png}
    \end{minipage}
    \caption{2D bar results. Partitioned mesh (left) and 100X warped displacement field (right).}
    \label{fig:5part}
\end{figure}

Note now in \cref{fig:5part} there are five subdomains in the
partitioned mesh since five cores were used. Contrary to previous
tutorial, as expected, we see that the right end of the bar which is
being pulled now contract in \(y\) direction. This is due to the fact
that there is no Dirichlet condition at this end now.
