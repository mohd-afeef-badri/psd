\newcommand{\psd}[1]{{\small\sffamily{\color{blue!60}#1}}}

Similar simulations, as in the previous tutorial is presented in this
section. We showcase the 2D bar problem simulation with one end clamped
wile being pulled at the other end. Contrary to simulation in the
previous tutorial, the clamped end just restricts \(x\) movement, i.e,
\(u_x=0\). Just like simulation from the previous tutorial the body
force is neglected. Just like simulation in the previous tutorial , the
non clamped ends pull is approximated with Neumann force
\(\int_{\partial\Omega^h_{\text N}}(\mathbf t\cdot \mathbf{v}^h)\). To
simulate the pull we assume traction vector
\(\mathbf t=[t_x,t_y]=[10^9,0]\) acting on the non clamped right end of
the bar, i.e., force in \(x\) direction is \(10^9\) units. The same
problem from previous tutorials 1 and 2 is used here, a bar 5 m in
length and 1 m in width, and is supposed to be made up of a material
with density \(\rho=8\times 10^3\), Youngs modulus \(E=200\times 10^9\),
and Poissons ratio \(\nu=0.3\). Here is how PSD simulation of this case
can be performed.

\textbf{Step 1: Preprocessing}

First step in a PSD simulation is PSD preprocessing, at this step you
tell PSD what kind of physics, boundary conditions, approximations,
mesh, etc are you expecting to solve.

In the terminal \psd{cd} to the folder
\psd{/home/PSD-tutorials/linear-elasticity}. Launch
\psd{PSD\_PreProcess} from the terminal, to do so run the following
command.

\begin{lstlisting}[style=BashInputStyle]
PSD_PreProcess -problem linear-elasticity -dimension 2 -dirichletconditions 1 -tractionconditions 1 \
-dirichletpointconditions 1 -postprocess u
\end{lstlisting}

Additional flag \psd{ -dirichletpointconditions 1} now appears, this
notifies to PSD that there is one Dirichlet point boundary condition.
Edit the \psd{ ControlParameters.edp} to communicate the desired point
boundary conditions, set the variables \psd{ Pbc0Ux  0.} and
\psd{ Pbc0Uy  0.} to specify \(u_x=0,u_y=0\), and variable
\psd{ PbcCord = [[  0. , 0. ]];} to specify the point coordinates
\((x,y)=(0,0)\). Via the flags we specified that
\psd{ -dirichletconditions 1}, i.e., there is one Dirichlet border. To
provide the Dirichlet condition (\(u_x=0\)) set the variables
\psd{ Dbc0On 2} and \psd{ Dbc0Ux 0.} in \psd{ ControlParameters.edp}.
PSD understands that 4 is the mesh border label on which Dirichlet is
applied and (\(u_x=0\)) is the condition to be applied.

\textbf{Step 2: Solving}

Let us now use 6 cores to solve this problem. To do so enter the
following command:

\begin{lstlisting}[style=BashInputStyle]
PSD_Solve -np 6 Main.edp -mesh ./../Meshes/2D/bar.msh -v 0
\end{lstlisting}

Notice, that this is the exact same command used in solving the previous
bar problems from other sections, with only difference that we now use
\psd{ -np 6}.

Note that for this simple problem, the bar mesh (\psd{bar.msh}) has been
provided in \psd{../Meshes/2D/"} folder, this mesh is a triangular mesh
produced with Gmsh. Moreover detailing meshing procedure is not the
propose of PSD tutorials. A user has the choice of performing their own
meshing step and providing them to PSD in
\psd{.msh}\footnote{Please use version 2} or \psd{.mesh} format, we
recommend using Salome or Gmsh meshers for creating your own geometry
and meshing them.

\textbf{Step 3: Postprocessing}

Launch ParaView and have a look at the \psd{ .pvd} file in the
\psd{ PSD/Solver/VTUs\_DATE\_TIME} folder.

\begin{figure}[htbp]
    \centering
    \begin{minipage}[t][2cm][t]{0.36\textwidth}
    \includegraphics[align=b,width=1\textwidth]{./Images/2d-bar-partitioned6.png}
    \end{minipage}\hspace{.1\textwidth}
    \begin{minipage}[t][2cm][t]{0.5\textwidth}
    \includegraphics[align=b,width=1\textwidth]{./Images/2d-bar-clamped-traction-point.png}
    \end{minipage}
    \caption{2D bar results. Partitioned mesh (left) and 100X warped displacement field (right).}
    \label{fig:6part}
\end{figure}

Note now in \cref{fig:6part} there are six subdomais in the partitioned
mesh. As expected, we see that the right and the left end of the bar
which is being pulled now contract in \(y\) direction, and the bar
elongates in \(x\) direction.
